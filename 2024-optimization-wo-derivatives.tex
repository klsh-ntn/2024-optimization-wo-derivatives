% arara: xelatex: {shell: yes}
% %arara: biber
% %arara: xelatex: {shell: yes}
% %arara: xelatex: {shell: yes}

\documentclass[12pt]{article}
\usepackage{libertine}

\usepackage{hyperref} % гиперссылки

\usepackage{tikz} % картинки в tikz
\usetikzlibrary{arrows.meta} % tikz-прибамбас для рисовки стрелочек подлиннее

\usepackage{microtype} % свешивание пунктуации

\usepackage{array} % для столбцов фиксированной ширины

\usepackage{indentfirst} % отступ в первом параграфе

\usepackage{sectsty} % для центрирования названий частей
\allsectionsfont{\centering}

\usepackage{amsmath} % куча стандартных математических плюшек
\usepackage{amssymb} % символы
\usepackage{amsthm} % теоремки
\usepackage{mathtools}

\usepackage{comment} % добавление длинных комментариев

\usepackage[top=2cm, left=1.2cm, right=1.2cm, bottom=2cm]{geometry} % размер текста на странице

\usepackage{lastpage} % чтобы узнать номер последней страницы

\usepackage{enumitem} % дополнительные плюшки для списков
%  например \begin{enumerate}[resume] позволяет продолжить нумерацию в новом списке

\usepackage{caption} % что-то делает с подписями рисунков :)

% \usepackage{qcircuit} % для рисовки квантовых диаграмм
% \usepackage{physics} % бракеты

\usepackage{answers} % разделение условий и ответов в упражнениях


\usepackage{fancyhdr} % весёлые колонтитулы
\pagestyle{fancy}
\lhead{Оптимизация без производных}
\chead{}
\rhead{КЛШ-2024 (47 сезон)}
\lfoot{}
\cfoot{}
\rfoot{\thepage/\pageref{LastPage}}
\renewcommand{\headrulewidth}{0.4pt}
\renewcommand{\footrulewidth}{0.4pt}



\usepackage{todonotes} % для вставки в документ заметок о том, что осталось сделать
% \todo{Здесь надо коэффициенты исправить}
% \missingfigure{Здесь будет Последний день Помпеи}
% \listoftodos — печатает все поставленные \todo'шки



\usepackage{booktabs} % красивые таблицы
% заповеди из докупентации:
% 1. Не используйте вертикальные линни
% 2. Не используйте двойные линии
% 3. Единицы измерения - в шапку таблицы
% 4. Не сокращайте .1 вместо 0.1
% 5. Повторяющееся значение повторяйте, а не говорите "то же"



\usepackage{fontspec} % что-то про шрифты?
\usepackage{polyglossia} % русификация xelatex

\setmainlanguage{russian}
\setotherlanguages{english}

% download "Linux Libertine" fonts:
% http://www.linuxlibertine.org/index.php?id=91&L=1
% \setmainfont{Linux Libertine O} % or Helvetica, Arial, Cambria
% why do we need \newfontfamily:
% http://tex.stackexchange.com/questions/91507/
% \newfontfamily{\cyrillicfonttt}{Linux Libertine O}

\AddEnumerateCounter{\asbuk}{\russian@alph}{щ} % для списков с русскими буквами
\setlist[enumerate, 2]{label=\asbuk*),ref=\asbuk*}

%% эконометрические сокращения
\DeclareMathOperator{\Cov}{Cov}
\DeclareMathOperator{\Arg}{Arg}
\DeclareMathOperator{\Corr}{Corr}
\DeclareMathOperator{\Var}{Var}
\DeclareMathOperator{\E}{\mathbb{E}}

\let\P\relax
\DeclareMathOperator{\P}{\mathbb{P}}
\newcommand{\const}{\mathrm{const}}

\usepackage{multicol}

\usepackage[bibencoding = auto,
backend = biber,
sorting = none,
style=alphabetic]{biblatex}

% \addbibresource{forecast_everything.bib}



% делаем короче интервал в списках
\setlength{\itemsep}{0pt}
\setlength{\parskip}{0pt}
\setlength{\parsep}{0pt}




\Newassociation{sol}{solution}{solution_file}
% sol --- имя окружения внутри задач
% solution --- имя окружения внутри solution_file
% solution_file --- имя файла в который будет идти запись решений
% можно изменить далее по ходу
\Opensolutionfile{solution_file}[all_solutions]
% в квадратных скобках фактическое имя файла

% магия для автоматических гиперссылок задача-решение
\newlist{myenum}{enumerate}{3}
% \newcounter{problem}[chapter] % нумерация задач внутри глав
\newcounter{problem}[section]

\newenvironment{problem}%
{%
\refstepcounter{problem}%
%  hyperlink to solution
     \hypertarget{problem:{\thesection.\theproblem}}{} % нумерация внутри глав
     % \hypertarget{problem:{\theproblem}}{}
     \Writetofile{solution_file}{\protect\hypertarget{soln:\thesection.\theproblem}{}}
     %\Writetofile{solution_file}{\protect\hypertarget{soln:\theproblem}{}}
     \begin{myenum}[label=\bfseries\protect\hyperlink{soln:\thesection.\theproblem}{\thesection.\theproblem},ref=\thesection.\theproblem]
     % \begin{myenum}[label=\bfseries\protect\hyperlink{soln:\theproblem}{\theproblem},ref=\theproblem]
     \item%
    }%
    {%
    \end{myenum}}
% для гиперссылок обратно надо переопределять окружение
% это происходит непосредственно перед подключением файла с решениями



\theoremstyle{definition}
\newtheorem{definition}{Определение}



\begin{document}

\newcommand{\dayone}{
AM-GM неравенство
\[
\frac{a + b}{2} \geq \sqrt{ab}, \quad \frac{x_1 + x_2 + \ldots + x_n}{n} \geq \sqrt[n]{x_1 \cdot x_2 \cdots x_n}
\]

\begin{enumerate}
  \item $xyz \to \max$ при условии $x + y + z = 600$, $x \geq 0$, $y \geq 0$, $z \geq 0$.
  \item $xy \to \max$ при условии $2x + y = 600$, $x \geq 0$, $y \geq 0$.
  \item $xy \to \max$ при условии $2x + y = 400$, $x \geq 0$, $y \geq 0$.
  \item $x^2y \to \max$ при условии $x + y = 300$, $x \geq 0$, $y \geq 0$.
  \item $x^3y^5 \to \max$ при условии $6x + 7y = 200$, $x \geq 0$, $y \geq 0$.
  \item $a + b + c \to \min$ при условии $abc = 100$, $a \geq 0$, $b \geq 0$, $c \geq 0$.
  \item $a^2 + b^2 + c^2 \to \min$ при условии $abc = 100$, $a \geq 0$, $b \geq 0$, $c \geq 0$.
  \item $ab + bc + ac \to \min$ при условии $abc = 100$, $a \geq 0$, $b \geq 0$, $c \geq 0$.
  \item $2a + 3b + 4c \to \min$ при условии $abc = 100$, $a \geq 0$, $b \geq 0$, $c \geq 0$.
  \item $2ab + 3bc + 4ac \to \min$ при условии $abc = 100$, $a \geq 0$, $b \geq 0$, $c \geq 0$.
  \item $a + b + c \to \min$ при условии $a^2 b^3 c^4 = 100$, $a \geq 0$, $b \geq 0$, $c \geq 0$.
  \item $7a + 3b + 4c \to \min$ при условии $a^2 b^3 c^4 = 100$, $a \geq 0$, $b \geq 0$, $c \geq 0$.
 \end{enumerate}
}



\newpage

\dayone

\vfill

\dayone

\section{Лог. КЛШ-2024}

\begin{enumerate}
  \item 
\end{enumerate}

В теховском файле \verb|\newpage| стоит, чтобы легко было скопировать секцию, для печати двух копий подряд на одном листе.
Это позволяет экономить бумагу и время при печати :)

\subsection{Плакат}





\Closesolutionfile{solution_file}

% для гиперссылок на условия
% http://tex.stackexchange.com/questions/45415
\renewenvironment{solution}[1]{%
         % add some glue
         \vskip .5cm plus 2cm minus 0.1cm%
         {\bfseries \hyperlink{problem:#1}{#1.}}%
}%
{%
}%



\section{Решения}
\input{all_solutions}


\section{Источники мудрости}

\printbibliography[heading=none]


\end{document}
